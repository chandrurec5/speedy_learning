\section{Synthesis of New algorithms}
To illustrate the usefulness of the ODE based framework, consider the following RL algorithm called SQLFA which combines SQL and function approximation for the purpose of on-policy evaluation:
\begin{align}\label{sqlfa}
\begin{split}
&\theta_{t+1}=\theta_t+\phi(s_t)^\top \alpha_t\big(R(s_t,a_t)+\gamma \phi(s_{t+1})\theta_{t-1}\\&-\phi(s_t)\theta_t +(1-\alpha_t)(\gamma \phi(s_{t+1})\theta_t-\gamma \phi(s_{t+1})\theta_{t-1}\big)
\end{split}
\end{align}
The ODE corresponding to SQLFA is given by
\begin{align}
\begin{split}
&\dot{\theta}(t)=(b_\pi-A_\pi\theta_t)+\gamma\Phi^\top D_\pi P_\pi\Phi^\top\dot{\theta}(t)\\
&\dot{\theta}(t)=(I-\gamma\Phi^\top D_\pi P_\pi\Phi)^{-1}(b_\pi-A_\pi\theta_t)
\end{split}
\end{align}

\begin{align}
\begin{split}
\theta_t&=\theta_t+\alpha_t(y_t-\phi(s_t)^\top(\phi(s_t)\theta_t-\gamma \phi(s_{t+1})\theta_t))\\
y_{t+1}&=y_{t}+\alpha_t(\delta_t)
\end{split}
\end{align}
The design corresponding to the above is given by $g_3=\begin{bmatrix}0\\ b_\pi\end{bmatrix}$, and $H_3=\begin{bmatrix} -A_\pi & I \\ -A_\pi & 0\end{bmatrix}$. The Eigen values of this design can be found by solving the following equations
\begin{align}
|\Lambda I-H|=\begin{bmatrix} \Lambda+A_\pi & -I \\ A_\pi & \Lambda\end{bmatrix}=(\Lambda+A_\pi)\Lambda+A_\pi=0
\end{align}
Thus if $\mu$ is a real Eigen value of $A_\pi$ and $\mu’_{1,2}$ corresponding Eigen values of $H_3$, we have the relation
\begin{align}
\mu’_{i}=\frac{\mu\pm\sqrt{\mu^2-4\mu}}{2}, i=1,2
\end{align}
Notice that for small values of $\mu$, $\mu’_i, i=1,2$ have imaginary parts and the real part is $\mu/2$. Thus this new scheme will be oscillatory (due to imaginary parts) and have poor convergence compared to \tdo since the real part gets divided by a factor of $2$.
