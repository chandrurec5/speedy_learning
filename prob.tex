\section{Problem Setup}
We consider Linear Stochastic Approximation (LSA) algorithms. An LSA algorithm with design $\D=\langle g,H\rangle$ in $n$ variables is the following stochastic recursion:
\begin{align}\label{linearrec}
x_{t+1}=x_t+\alpha_t(g-H_t x_t)+\alpha_t N_{t+1}.
\end{align}
The assumptions on the variables in \eqref{linearrec} are as under
\begin{assumption}\label{lsaassump}
\hspace{10pt}\\
\vspace{-20pt}
\begin{enumerate}[leftmargin=*]
\item $x_t\in \R^n$ are the iterates and $\alpha_t\geq 0 ,\forall t\geq 0$ are the step-sizes.
\item\label{pd} $H_t\in \R^{n\times n}$ are i.i.d matrices with $\E[H_t]=H$, where $H$ is a real positive definite design matrix.
\item $g\in \R^n$ is the design vector.
\item $N_{t+1}$ are martingale difference term such that $\E[\parallel N_{t+1} \parallel^2]\leq \sigma^2$ for some variance $\sigma\in \R^+$.
\end{enumerate}
\end{assumption}
\subsection{ODE and Asymptotic Results}
\begin{theorem}[Asymptotics]\label{linstab} When the step-size rule $\{\alpha_t,t\geq 0\}$ is such that $\sum_{t\geq 0} alpha_t =\infty$ and $\sum_{t\geq 0} alpha^2_t <\infty$, then
\begin{enumerate}[leftmargin=*] 
\item Let $x(s),s\geq 0$ denote the solution to the ordinary differential equation (ODE) $\dot{x}(s)=g-Hx(s)$, with $x(0)=x_0$ and let $s(t)=\sum_{0\leq k< t}\alpha_k$. Under the above assumptions we have iterates $x_t\ra x(s(t))$ as $t\ra\infty$.
\item The iterates in the linear stochastic update rule in \eqref{linearrec} converge to $x^*=H^{-1}g$ the unique asymptotic stable equilibrium of the ODE $\dot{x}(s)=g-Hx(s)$, i.e., $x_t\ra H^{-1}g$ as $t\ra\infty$.
\end{enumerate}
\end{theorem}
\begin{proof}
Follows from arguments in Chapter~$2$, \cite{borkarsa}.
\end{proof}
The important take away from \Cref{linstab} is that for diminishing step-sizes the LSA follows the trajectory of the ODE and it converges to the desired solution when $H$ is positive definite. We now illustrate the role played by the step-size rule in the rate of convergence of the LSA algorithm by looking at an approximate expression for forgetting the initial condition $x_0$. Suppose, the design matrix $H$ has all positive Eigen values given by $\{\mu_i,i=1,\ldots,n\}$, then it is known from standard results in linear system theory \cite{} that the trajectory $x(s),s\geq 0$ of the ODE $\dot{x}(s)=(g-Hx(s))$ with $x(0)=x_0$ is given by
\begin{align}\label{oderate}
x(s)=\sum_{i=1}^n \zeta_i e^{-\mu_i s}, 
\end{align}
where $\zeta=(\zeta_i,i=1,\ldots,n)\in \R^n$ are real coefficients. The time $s\geq 0$ in \eqref{oderate} is \emph{real} time and the time corresponding to the $t^{th}$ iterate of the algorithm is roughly $s(t)\approx\sum_{0\leq k<t}\alpha_t$. It is easy to see from \eqref{oderate} that the rate of forgetting initial conditions depends on the Eigen values values and the accumulation of algorithm time. For instance, if the step-size are chosen to be $\alpha_t=C/t$, then \begin{align}\label{biasforget}e^{-\mu_i\sum_{0\leq k<t}\alpha_t}\approx e^{-\mu_i Clog s}=O(1/s^{\mu_i C})\end{align}
It is clear that the forgetting the initial condition depends both on the step-size rule which dictates accumulation of time i.e., $\sum_{0\leq k<t}\alpha_t$ and the smallest Eigen value of the design matrix. 
\subsection{Main Results: Finite Time Performance Bounds}
We now present the finite time performance of the LSA algorithms by deriving bounds $\E[\parallel x_n-x^*\parallel^2]$. The bound involves two terms namely the \emph{bias} term due to the initial condition $x_0$ and the variance term due to the noise term. 
\begin{lemma}\label{alphacond}
For $H_t, H$ as in condition~\ref{pd} of \Cref{lsaassump}, there exists a $\alpha_{\max}>0$ such that $(H^\top+H)-\alpha (\E[H_t^\top H_t])\geq 0, \forall 0< \alpha<\alpha_{\max}$ 
\end{lemma}
\begin{proof}
The proof is complete by noting that $H^\top+H$ and $\E[H^\top_t H]$ are real symmetric positive definite matrices.
\end{proof}
Let $0<\alpha<\alpha_{\max}$, then $\rho_{\alpha}\eqdef\parallel I-\alpha(H^\top+H)+\alpha^2\E[H^\top_t H_t]\parallel$ (where $\parallel\cdot\parallel$ is the spectral norm). The following theorem is similar to Theorem~$1$, \cite{bachaistats}.
\begin{theorem}[Finite Time Bound]\label{maintheorem}
Let $\bar{x}_t=\frac{1}{t}\sum_{i=0}^{t-1} x_t$ be the Rupert-Polyak average of the iterates of LSA algorithm in \eqref{linearrec}. Then for $0<\alpha< \alpha_{\max}$, we have
\begin{align}
&\E[\parallel \bar{x}_t-x^*\parallel^2]\leq \underbrace{\Big(\frac{2}{(1-\rho_{\alpha})^2t^2}\Big)\parallel x_0-x^*\parallel^2}_{\text{Bias}}\\&+\underbrace{\Big(\frac{1}{(1-\rho_{\alpha})t}\Big)\alpha^2\sigma^2}_{\text{Variance}}
\end{align}
\end{theorem}
\begin{proof}
Let for $t> k $, $F_{t,k}=\prod_{i=k}^t(I-\alpha_i H_i)$ and for $t\leq k$, $F_{t,k}=I$. Further, we have ${x}_t-x^*=F_{t,i} (x_i-x_0)+\sum_{k=i}^{t}\alpha_i F_{t,k+1}N_{i+1}$
\begin{align*}
&\E[\parallel\bar{x}_t-x^*\parallel^2]=\E\parallel\frac{1}{t}\sum_{i=0}^{t-1} (x_i-x^*)\parallel^2\\
&=\frac{1}{t^2}\E\sum_{i=0}^{t-1} \sum_{j=0}^{t-1}(x_i-x^*)^\top(x_j-x^*)\\
%&=\frac{1}{t^2}\E\sum_{i=0}^{t-1}\big((x_i-x^*)F^\top_{i,0} F_{i,0}(x_i-x^*)\\&+ \sum_{j=i+1}^{t-1}(x_i-x^*)^\top F^\top_{j,i}F^\top_{j,i} (x_i-x^*)\\&+ \sum_{j=0}^{i-1}(x_0-x^*)^\top F^\top_{j,0}F^%\top_{j,0} (x_0-x^*)\big)\\
&=\textbf{Bias}+\textbf{Variance}
\end{align*}
Let $z^*\eqdef x^*, z_0\eqdef x_0$, and $z_t-z^*\eqdef F_{t,0}(x_0-x^*)$, then we have
\begin{align*}
&\textbf{Bias}=\frac{1}{t^2}[\sum_{i=0}^{t-1}\E[(z_i-z^*) (z_i-z^*)]\\&+2\sum_{i=0}^{t-1}\sum_{j=i+1}^{t-1}(z_i-z^*)^\top F_{j,i+1} (z_i-z^*)]\\
&\leq \frac{1}{t^2}[\sum_{i=0}^{t-1}\E[2(z_i-z^*)(z_i-z^*)]\\&+2\sum_{i=0}^{t-1}\sum_{j=i+1}^{t-1}(z_i-z^*)^\top \rho_\alpha^{j-i} (z_i-z^*)]\\
&\leq \frac{1}{t^2}[\sum_{i=0}^{t-1}\E[\frac{2}{1-\rho_\alpha}(z_i-z^*)(z_i-z^*)]\\
&=\frac{1}{t^2}[\sum_{i=0}^{t-1}\E[\frac{2}{1-\rho_\alpha}(z_0-z^*)F^\top_{i,0}F_{i,0}(z_0-z^*)]\\
&\leq \frac{2}{(1-\rho_{\alpha})t^2}\frac{1}{1-\rho_\alpha}\E(z_0-x^*)\rho_\alpha^i(z_0-x^*)]\\
\end{align*}
\begin{align*}
&\textbf{Variance}=\frac{1}{t^2}\E[\sum_{i=0}^{t-1}\sum_{j=i}^{t-1} \alpha^2 N^\top_{i+1}F^\top_{j,i} F_{j,i} N_{i+1}]\\
&\leq \frac{1}{t^2}\E[\sum_{i=0}^{t-1}N_{i+1}^\top N_{i+1}\sum_{j=i}^{t-1} \alpha^2 \rho_{\alpha}^2]\\
&\leq \frac{\alpha^2}{t^2}[\sum_{i=0}^{t-1}\sigma^2 \frac{1}{1-\rho_{\alpha}}]\\
&\leq \frac{\alpha^2\sigma^2}{(1-\rho_\alpha)t}
\end{align*}
\end{proof}
\input{opti.tex}