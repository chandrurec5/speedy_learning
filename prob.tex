\section{Problem Setup}
A linear stochastic approximation (LSA) algorithm with design $\D=\langle g,H\rangle$ in $n$ variables is given by the following stochastic recursion:
\begin{align}\label{linearrec}
x_{t+1}=x_t+\alpha_t(g-H_t x_t)+\alpha_t N_{t+1}.
\end{align}
The assumptions on the quantities in \eqref{linearrec} are as under.
\begin{assumption}\label{lsaassump}
\hspace{10pt}\\
\vspace{-20pt}
\begin{enumerate}[leftmargin=*]
\item $x_t\in \R^n$ are the iterates and $\{\alpha_t\geq 0 ,\forall t\geq 0\}$ is the step-size rule.
\item\label{pd} $H_t\in \R^{n\times n}$ are i.i.d matrices with $\E[H_t]=H$, where $H$ is a real positive definite design matrix.
\item $g\in \R^n$ is the design vector.
\item $N_{t+1}$ are martingale difference terms  with respect to an increasing family of $\sigma$-fields $\mathcal{F}_t\stackrel{\cdot}{=}\sigma(x_0,N_1,\ldots,N_t),t\geq 0$, such that $\E[\parallel N_{t+1} \parallel^2]\leq \sigma^2$ for some variance $\sigma>0$.
\end{enumerate}
\end{assumption}
\subsection{ODE and Asymptotic Results}\label{asymp}
\begin{theorem}[Asymptotics]\label{linstab} When the step-size rule $\{\alpha_t,t\geq 0\}$ is such that $\sum_{t\geq 0} \alpha_t =\infty$ and $\sum_{t\geq 0} \alpha^2_t <\infty$, then
\begin{enumerate}[leftmargin=*] 
\item Let $x(s),s\geq 0$ denote the solution to the ordinary differential equation (ODE) $\dot{x}(s)=g-Hx(s)$, with $x(0)=x_0$ and let $s(t)=\sum_{0\leq k< t}\alpha_k$. Under \Cref{lsaasump} we have iterates $x_t\ra x(s(t))$ as $t\ra\infty$.
\item Let $x^*=H^{-1}g$ be the unique asymptotic stable equilibrium of the ODE $\dot{x}(s)=g-Hx(s)$. Then the iterates in the LSA algorithm in \eqref{linearrec} converge to $x^*$ i.e., $x_t\ra H^{-1}g$ as $t\ra\infty$.
\end{enumerate}
\end{theorem}
\begin{proof}
Follows from arguments in Chapter~$2$, \cite{borkarsa}.
\end{proof}
The important take away from \Cref{linstab} is that for diminishing step-sizes the iterates $x_t$ of the LSA algorithm follow the trajectory of a corresponding ODE and they converge to the desired solution when $H$ is positive definite. We now illustrate the role played by the step-size rule in the rate of convergence of the LSA algorithm by looking at an approximate expression for forgetting the initial condition $x_0$. Suppose, the design matrix $H$ has all positive Eigen values given by $\{\mu_i,i=1,\ldots,n\}$, then it is known from standard results in linear system theory \cite{chen} that the trajectory $x(s),s\geq 0$ of the ODE $\dot{x}(s)=(g-Hx(s))$ with $x(0)=x_0$ is given by
\begin{align}\label{oderate}
x(s)=\sum_{i=1}^n \zeta_i e^{-\mu_i s}, 
\end{align}
where $\zeta=(\zeta_i,i=1,\ldots,n)\in \R^n$ are real coefficients. The time $s\geq 0$ in \eqref{oderate} is \emph{real} time and the time corresponding to the $t^{th}$ iterate of the LSA algorithm is roughly $s(t)\approx\sum_{0\leq k<t}\alpha_t$. It is easy to see from \eqref{oderate} that the rate of forgetting initial conditions depends on the Eigen values and the accumulation of the algorithm time. For instance, if the step-size rule is chosen to be $\alpha_t=C/t$, then \begin{align}\label{biasforget}e^{-\mu_i\sum_{0\leq k<t}\alpha_t}\approx e^{-\mu_i Clog s}=O(1/s^{\mu_i C})\end{align}
It is clear that the forgetting of the initial condition depends both on the step-size rule which dictates accumulation of time i.e., $\sum_{0\leq k<t}\alpha_t$ and the Eigen values (error term due to the smallest Eigen value dominates the other terms) of the design matrix. 
\subsection{Main Results: Finite Time Performance Bounds}\label{main result}
We now present the finite time performance of the LSA algorithm by deriving bounds for $\E[\parallel x_n-x^*\parallel^2]$. The bound involves two terms namely the \emph{bias} term due to the initial condition $x_0$ and the variance term due to the noise term. 
\begin{lemma}
For $H_t, H$ as in condition~\ref{pd} of \Cref{lsaassump}, there exists a $\alpha_{\max}>0$ such that $(H^\top+H)-\alpha (\E[H_t^\top H_t])\geq 0, \forall 0< \alpha<\alpha_{\max}$ 
\end{lemma}
\begin{proof}
The proof is complete by noting that $H^\top+H$ and $\E[H^\top_t H]$ are real symmetric positive definite matrices.
\end{proof}
Let $0<\alpha<\alpha_{\max}$, then $\rho_{\alpha}\eqdef\parallel I-\alpha(H^\top+H)+\alpha^2\E[H^\top_t H_t]\parallel$ (where $\parallel\cdot\parallel$ is the spectral norm). The following theorem is similar to 
\begin{theorem}
Let $\bar{x}_t=\frac{1}{t}\sum_{i=0}^{t-1} x_t$ be the Rupert-Polyak average of the iterates of LSA algorithm in \eqref{linearrec}. Then for $0<\alpha< \alpha_{\max}$, we have
\begin{align}
&\E[\parallel \bar{x}_t-x^*\parallel^2]\leq \underbrace{\Big(\frac{2}{(1-\rho_{\alpha})^2t^2}\Big)\parallel x_0-x^*\parallel^2}_{\text{Bias}}\\&+\underbrace{\Big(\frac{1}{(1-\rho_{\alpha})t}\Big)\alpha^2\sigma^2}_{\text{Variance}}
\end{align}
\end{theorem}
\begin{proof}
Let for $t> k $, $F_{t,k}=\prod_{i=k}^t(I-\alpha_i H_i)$ and for $t\leq k$, $F_{t,k}=I$. Further, we have ${x}_t-x^*=F_{t,i} (x_i-x_0)+\sum_{k=i}^{t}\alpha_i F_{t,k+1}N_{i+1}$
\begin{align*}
&\E[\parallel\bar{x}_t-x^*\parallel^2]=\E\parallel\frac{1}{t}\sum_{i=0}^{t-1} (x_i-x^*)\parallel^2\\
&=\frac{1}{t^2}\E\sum_{i=0}^{t-1} \sum_{j=0}^{t-1}(x_i-x^*)^\top(x_j-x^*)\\
%&=\frac{1}{t^2}\E\sum_{i=0}^{t-1}\big((x_i-x^*)F^\top_{i,0} F_{i,0}(x_i-x^*)\\&+ \sum_{j=i+1}^{t-1}(x_i-x^*)^\top F^\top_{j,i}F^\top_{j,i} (x_i-x^*)\\&+ \sum_{j=0}^{i-1}(x_0-x^*)^\top F^\top_{j,0}F^%\top_{j,0} (x_0-x^*)\big)\\
&=\textbf{Bias}+\textbf{Variance}
\end{align*}
Let $z^*\eqdef x^*, z_0\eqdef x_0$, and $z_t-z^*\eqdef F_{t,0}(x_0-x^*)$, then we have
\begin{align*}
&\textbf{Bias}=\frac{1}{t^2}[\sum_{i=0}^{t-1}\E[(z_i-z^*) (z_i-z^*)]\\&+2\sum_{i=0}^{t-1}\sum_{j=i+1}^{t-1}(z_i-z^*)^\top F_{j,i+1} (z_i-z^*)]\\
&\leq \frac{1}{t^2}[\sum_{i=0}^{t-1}\E[2(z_i-z^*)(z_i-z^*)]\\&+2\sum_{i=0}^{t-1}\sum_{j=i+1}^{t-1}(z_i-z^*)^\top \rho_\alpha^{j-i} (z_i-z^*)]\\
&\leq \frac{1}{t^2}[\sum_{i=0}^{t-1}\E[\frac{2}{1-\rho_\alpha}(z_i-z^*)(z_i-z^*)]\\
&=\frac{1}{t^2}[\sum_{i=0}^{t-1}\E[\frac{2}{1-\rho_\alpha}(z_0-z^*)F^\top_{i,0}F_{i,0}(z_0-z^*)]\\
&\leq \frac{2}{(1-\rho_{\alpha})t^2}\frac{1}{1-\rho_\alpha}\E(z_0-x^*)\rho_\alpha^i(z_0-x^*)]\\
\end{align*}
\begin{align*}
&\textbf{Variance}=\frac{1}{t^2}\E[\sum_{i=0}^{t-1}\sum_{j=i}^{t-1} \alpha^2 N^\top_{i+1}F^\top_{j,i} F_{j,i} N_{i+1}]\\
&\leq \frac{1}{t^2}\E[\sum_{i=0}^{t-1}N_{i+1}^\top N_{i+1}\sum_{j=i}^{t-1} \alpha^2 \rho_{\alpha}^2]\\
&\leq \frac{\alpha^2}{t^2}[\sum_{i=0}^{t-1}\sigma^2 \frac{1}{1-\rho_{\alpha}}]\\
&\leq \frac{\alpha^2\sigma^2}{(1-\rho_\alpha)t}
\end{align*}
\end{proof}
\subsection{Maximum Allowable Step-Size}\label{opti}
The condition that $0<\alpha<\alpha_{\max}$ ensures that $\rho_{\alpha}<1$ and in \cite{bachaistats} authors conjecture that this bound on $\alpha_{\max}$ is strict i.e., there exists some initial condition $x_0$ for which the LSA in \eqref{linearrec} is unstable. We now present a theorem from \cite{logexp} and simple counter examples to falsify this conjecture.
\begin{theorem}\label{explog}
Let $\mathcal{H}=(H_t), t\geq 0$ be a stationary process of $n\times n$ real-valued matrices over some probability space $(\Omega,\F,\mathcal{P})$. If $\E\log^+\parallel H_0\parallel<\infty$ (where $\log^+ x$ denotes the positive part of $\log x$), then there exists a $\lambda\in \R$ that satisfies
\begin{align}\label{lambda}
\lambda=\lim_{t}\frac{1}{t}\log\parallel H_t H_{t-1}\ldots H_0\parallel
\end{align}
\end{theorem}
In the case when the implicit relation in \eqref{lambda} yields a $\lambda<0$, it is also implied that $\{z_t\},t\geq 0$ such that $z_t=H_t H_{t-1}\ldots H_0z_0$ is stable.
\begin{example}[Linear Prediction]
Let the data represented as $(input,output)$ be $(X_t,Y_t)\in \{(2,0), (4,0)\}$ with equal probability. The problem of linear prediction is then to find $\theta^*\in R$ such that it minimizes the loss $\E(X_t \theta^* -Y_t)^2$, and the SGD algorithm to solve find $\theta^*$ is given by
\begin{align}
\theta_{t+1}=\theta_t+\alpha(y_t X_t-X_t\otimes X_t\theta_t),
\end{align}
where $\alpha$ is the constant step-size. Now the condition on $\alpha_{\max}$ presented in \Cref{alphacond} and in \cite{bachaistats}, translates to the following numerical condition in this specific example
\begin{align*}
\alpha_{\max}<\frac{2H}{\E[H_t^\top H_t]}=\frac{10}{17}
\end{align*}
We now derive $\alpha^{\lambda}_{\max}$ which is the maximum allowable constant step-size as suggested by the implicit relation in \eqref{lambda}. We have
\begin{align}
\lambda=\frac{1}{2}\log(1-\alpha)+\frac{1}{2}\log(1-\alpha 4).
\end{align}
For stability we need $\lambda<0$, i.e., $-1<(1-\alpha)(1-alpha 4)<1$, which translates to the condition $0<\alpha<\alpha^{\lambda}_{\max}=\frac{5}{4}$. It is clear that $\alpha^{\lambda}_{\max}>\frac{2}{\E[H]}=0.8>\alpha_{\max}$.
\end{example}
A similar counter example can be provided in the asymmetric case as follows
\begin{example}[General LSA]
Consider the LSA in \eqref{linearrec} with $x_t\in \R$ and $H_t\sim \{-1, 2\}$ (with equal probability)  and $g_t=0$. Then 
$
\alpha_{\max}<\frac{2H}{\E[H_t^\top H_t]}=\frac{2}{5}
$
and since 
$
\lambda=\frac{1}{2}\log(1+\alpha)+\frac{1}{2}\log(1-\alpha 2).
$
we have $\alpha^{\lambda}_{\max}=0.78078$. It is clear that $\alpha^{\lambda}_{\max}>\alpha_{\max}$. However, in the asymmetric case we have $\frac{2}{\E[H]}=1>\alpha^{\lambda}_{\max}>\alpha_{\max}$.
\end{example}
\FloatBarrier
\begin{figure}[htp]
\begin{minipage}{0.5\textwidth}
\resizebox{1.0\textwidth}{!}{
\begin{tabular}{cc}
\begin{tikzpicture}[scale=1,font=\Large,]
    \begin{axis}[
        xlabel=$t$,
        ylabel=$\theta_t$,legend style={at={(0.5,-0.1)},anchor=north}
]

    \addplot[only marks,mark=square,red] plot file {./experiments/symm_stable_samp};
    \addplot[only marks,mark=diamond,blue] plot file {./experiments/asymm_stable_samp};

\addlegendentry{{\color{black}{Example~1,$\alpha=1.1$}}}
\addlegendentry{{\color{black}{Example~2,$\alpha=0.7$}}}


    \addplot[thick,dashed,mark=.,red] plot file {./experiments/symm_stable};
    \addplot[thick,dashed,mark=.,blue] plot file {./experiments/asymm_stable};

    \end{axis}
    \end{tikzpicture}

&
\begin{tikzpicture}[scale=1,font=\Large]
    \begin{axis}[
        xlabel=$t$,
        ylabel=$\theta_t$,legend style={at={(0.5,-0.1)},anchor=north}
]

    \addplot[only marks,mark=square,red] plot file {./experiments/symm_unstable_samp};
    \addplot[only marks,mark=diamond,blue] plot file {./experiments/asymm_unstable_samp};

\addlegendentry{{\color{black}{$\theta_t$ Example~1,$\alpha=1.3$}}}
\addlegendentry{{\color{black}{$\theta_t\times 10^{25}$Example~2,$\alpha=0.8$}}}


    \addplot[thick,dashed,mark=.,red] plot file {./experiments/symm_unstable};
    \addplot[thick,dashed,mark=.,blue] plot file {./experiments/asymm_unstable};

    \end{axis}
    \end{tikzpicture}

\end{tabular}
}
\end{minipage}
\end{figure}


